%%%%%%%%%%%%%%%%%%%%%%%%%%%%%%%%%%%%%%%%%%%%%%%%%%%%%%%%%
% Niniejszy plik przedstawia przykładowy skład 
% pracy dyplomowej na Wydziale Matematyki PWr. 
% 
% Autorzy: 
% Damian Fafuła
% Michał Kijaczko
% Jakub Michalczak
% Maciej Miśta
% Dagmara Nowak
% Tomasz Skalski
% Wojciech Słomian
%
%% Data utworzenia: 8.05.2018
% Numer wersji: 1
%
% Poniższą formatkę można rozpowszechniać i edytować 
% pod warunkiem zachowania numeru wersji, 
% informacji o autorach i dodaniu informacji 
% o wprowadzonych zmianach.
%
%%%%%%%%%%%%%%%%%%%%%%%%%%%%%%%%%%%%%%%%%%%%%%%%%%%%%%%%%
% Domyślną opcją jest: praca magisterska, język polski.
% W przypadku pracy pisanej w języku angielskim dodajemy 
% opcję [english].
% Dla pracy licencjackiej dodajemy opcję [licencjacka].
% Dla pracy inżynierskiej dodajemy opcję [inzynierska].
% Dopuszczalne są podwójne opcje, np. [licencjacka, english].
% Opcje dodajemy w kwadratowym nawiasie przy \documentclass.
%
%
%%%%%%%%%%%%%%%%%%%%%%%%%%%%%%%%%%%%%%%%%%%%%%%%%%%%%%%%%
\documentclass[inzynierska]{pwr_wmat_praca_dyplomowa}
\usepackage{graphicx}
%%%%%%%%%%%%%%%%%%%%%%%%%%%%%%%%%%%%%%%%%%%%%%%%%%%%%%%%%
%              DANE DO PRACY
%
% W przypadku pracy dyplomowej w języku angielskim nie jest konieczne 
% wypełnianie pól: \tytul{}, \kierunek{}, \specjalnosc{}, 
%                  \streszczenie{}, \slowakluczowe{}.
%%%%%%%%%%%%%%%%%%%%%%%%%%%%%%%%%%%%%%%%%%%%%%%%%%%%%%%%%
%
% Imię i nazwisko autora
\autor{Jakub Rozenbajger}
%
% Tytuł pracy dyplomowej 
\tytul{Projekt i implementacja aplikacji webowej opartej o architekturę czysto funkcyjnych mikroserwisów.}
\tytulang{Design and implementation of a web application based on purely functional microservices architecture.}
%
% Tytuł / stopień / imię i nazwisko opiekuna
\opiekun{dr inż. Michał Szczepanik}
%
% Kierunek studiów wybieramy spośród następujących:
% 1) Matematyka
% 2) Matematyka i Statystyka
% 3) Matematyka stosowana
\kierunekstudiow{Informatyka}
%
% Kierunek studiów po angielsku wybieramy spośród następujących:
% 1) Mathematics
% 2) Mathematics and Statistics
% 3) Applied Mathematics
\kierunekstudiowang{Computer Science}
%
% Specjalność wybieramy spośród następujących: 
% KIERUNEK: Matematyka
% 1) Matematyka teoretyczna,
% 2) Statystyka matematyczna,
% 3) Matematyka finansowa i ubezpieczeniowa,
%
% KIERUNEK: Matematyka i Statystyka
% 4) Matematyka,
% 5) Statystyka i analiza danych, 
%
% 6) -- (w przypadku braku specjalizacji).
\specjalnosc{--}
%
% Specjalność w języku angielskim wybieramy spośród następujących:
% KIERUNEK: Matematyka
% 1) Theoretical Mathematics,
% 2) Mathematical Statistics,
% 3) Financial and Actuarial Mathematics,
%
% KIERUNEK: Matematyka i Statystyka
% 4) Mathematics,
% 5) Statistics and Data Analysis,
%
% KIERUNEK: Applied Mathematics
% 6) Financial and Actuarial Mathematics, 
% 7) Mathematics for Industry and Commerce,
% 8) Computational Mathematics,
% 9) Modelling, Simulation and Optimization.
%
% 10) -- (w przypadku braku specjalizacji).
\specjalnoscang{--}
%
% Krótkie streszczenia po polsku i angielsku
% - nie dłuższe niż 530 znaków.
\streszczenie{Zostanie rozwiązywany problem
oraz zostaną przedstawione rozwiązania konkurencyjne. Zostanie opracowana i przedstawiona
architektura aplikacji, a także zostanie dokonana analiza jej wpływu na kwestie utrzymania i niezawodności.
Trzeci rozdział będzie zawierał opis technologii wykorzystywanych w ramach pracy i ich wpływu na proces implementacyjny wraz z
zakończy jej podsumowanie, gdzie zostaną określone dalsze możliwości rozwoju.}
\streszczenieang{Zostanie rozwiązywany problem
oraz zostaną przedstawione rozwiązania konkurencyjne. Zostanie opracowana i przedstawiona
Trzeci rozdział będzie zawierał opis technologii wykorzystywanych w ramach pracy i ich wpływu na proces implementacyjny wraz z
projektu realizowanej aplikacji. W przedostatnim rozdziale zaprezentowana zostanie zrealizowana aplikacja. Pracę
zakończy jej podsumowanie, gdzie zostaną określone dalsze możliwości rozwoju.}
%
% Podajemy najważniejsze słowa kluczowe po polsku i angielsku
% - w obu przypadkach, nie więcej niż 150 znaków.
\slowakluczowe{programowanie funkcyjne mikroserwisy aplikacja webowa scala haskell}
\slowakluczoweang{functional programming microservices web app scala haskell}
%
%
%%%%%%%%%%%%%%%%%%%%%%%%%%%%%%%%%%%%%%%%%%%%%%%%%%%%%%%%%
% Definicje, lematy, twierdzenia, przykłady i wnioski
% Komendy wywołujące twierdzenia, definicje, itd., 
% czyli 'theorem', 'definition', 'corollary', itd., 
% można zmienić wedle uznania.
\theoremstyle{plain}
\newtheorem{theorem}{Twierdzenie}
\numberwithin{theorem}{chapter}
\newtheorem{lemma}[theorem]{Lemat}
\newtheorem{corollary}[theorem]{Wniosek}
\newtheorem{fact}[theorem]{Fakt}
\theoremstyle{definition}
\numberwithin{theorem}{chapter}
\newtheorem{definition}[theorem]{Definicja}
\newtheorem{example}[theorem]{Przykład}
\newtheorem{note}[theorem]{Uwaga}
%%%%%%%%%%%%%%%%%%%%%%%%%%%%%%%%%%%%%%%%%%%%%%%%%%%%%%%%%


%%%%%%%%%%%%%%%%%%%%%%%%%%%%%%%%%%%%%%%%%%%%%%%%%%%%%%%%%
%%%%%%%%%%%%%%%%%%%%%%%%%%%%%%%%%%%%%%%%%%%%%%%%%%%%%%%%%
\begin{document}
    \frontmatter
    \maketitle
    \mainmatter
    \tableofcontents
    %\listoffigures
    %\listoftables

    {\backmatter \chapter{Wstęp}}
    We wstępie zapowiadamy, o czym będzie praca. Próbujemy zachęcić czytelnika do dalszej lektury, np. krótko informując, dlaczego wybraliśmy właśnie ten temat i co nas w nim zainteresowało.

    \chapter{Rozdział pierwszy}
    Tabela \ref{tab:przykladowa} przedstawia przykładową tabelę. Do tworzenia tabeli służą m.in. środowiska \texttt{tabular} oraz \texttt{table}. Istnieje możliwość numeracji dwustopniowej, gdzie pierwsza cyfra oznacza numer rozdziału, a druga – kolejny numer tabeli w tym rozdziale. Tytuł powinien znajdować się centralnie nad tabelą, $12$ pkt odstępu od tekstu zasadniczego nad i pod tabelą wraz z tytułem. Jeśli tabela jest cytowana – należy podać centralnie pod tabelą źródło jej pochodzenia, np. opracowanie własne, opracowano na podstawie danych z GUS.
    \begin{table}[ht]
        \caption{Podstawowa Tabela}
        \centering
        \begin{tabular}{ccc}
            \hline
            \hline
            Państwo & PKB (w milionach USD )& Stopa bezrobocia  \\  [0.5ex]
            \hline
            Stany Zjednoczone & 75 278 049 & 4,60\%  \\
            Chiny & 11 218 281 & 4,10\%   \\
            Japonia & 4 938 644 & 3,10\%  \\
            Niemcy & 3 466 639 & 6,00\%   \\
            Wielka Brytania & 2 629 188 & 4,60\%  \\ [1ex]
            \hline
        \end{tabular}
        \caption*{\textit{Źródło: opracowanie własne}}
        \label{tab:przykladowa}
    \end{table}

    Do cytowania używamy komendy \texttt{cite}. W nawiasie klamrowym podajemy klucz, którego użyliśmy w pliku \emph{bibliografia.bib}. Przykład: \cite{einstein} lub \cite[chap. 2]{latexcompanion}.

    \section{Podrozdział pierwszy}

    \begin{table}[H]
        \caption{Podstawowa Tabela}
        \centering
        \begin{tabular}{ccc}
            \hline
            \hline
            Państwo & PKB (w milionach USD )& Stopa bezrobocia  \\  [0.5ex]
            \hline
            Stany Zjednoczone & 75 278 049 & 4,60\%  \\
            Chiny & 11 218 281 & 4,10\%   \\
            Japonia & 4 938 644 & 3,10\%  \\
            Niemcy & 3 466 639 & 6,00\%   \\
            Wielka Brytania & 2 629 188 & 4,60\%  \\ [1ex]
            \hline
        \end{tabular}
        \caption*{\textit{Źródło: opracowanie własne}}
        \label{tab:przykladowa2}
    \end{table}

    \section{Podrozdział drugi}

    Rysunki do pracy dyplomowej należy wstawiać w sposób podobny do wstawiania tabel, z~zasadniczą różnicą polegającą na tym, że podpis powinno umieszczać się centralnie pod rysunkiem, a nie powyżej niego. Numeracja i sposób cytowania pozostają bez zmian, przy czym tabele i rysunki nie mają numeracji wspólnej, np. po Tabeli \ref{tab:przykladowa2} występuje Rysunek \ref{rys1} (o ile jest to pierwszy rysunek rozdziału pierwszego), a nie Rysunek $1.3$.

    \begin{figure}[ht]

        \centering

        \includegraphics[scale=0.27]{logo_w13.jpg}
        \caption{Podstawowy Rysunek}\label{rys1}
    \end{figure}
    \label{rys:przykladowy}

    \chapter{Przegląd technologii}
    \section{Postgres}
    \section{Typescript}
    \section{Django}
    \section{Node.js}
    \section{Hapi.js}
    \section{Vue.js}
    \section{Angular}
    \section{React}
    \section{Docker}

    \chapter{Konkurencyjne rozwiązania}

    \chapter{Szczegółowy opis technologii}
    \chapter{Projekt}
    \section{Wymagania}
    \subsection{Funkcjonalne}
    \subsection{Niefunkcjonalne}
    \chapter{Implementacja}
    \section{Backend}
    \subsection{Django-like ORM}
    \section{Frontend}
    \chapter{Wrożenie}
    \chapter{Testy}




    \chapter{Definicje, lematy, twierdzenia, przykłady i wnioski}
    Definicje, lematy, twierdzenia, przykłady i wnioski piszemy w pracy tak:
    \begin{definition}[Martyngał]
        Tu piszemy treść definicji martyngału.
    \end{definition}
    \begin{lemma}[]% w nawiasie kwadratowym można napisać jego nazwę
        Tu piszemy treść lematu.
    \end{lemma}

    {\backmatter \chapter{Podsumowanie}}
    Podsumowanie w pracach matematycznych nie jest obligatoryjne. Warto jednak na zakończenie krótko napisać, co udało nam się zrobić w pracy, a czasem także o tym, czego nie udało się zrobić.

    {\backmatter \chapter{Dodatek}}
    Dodatek w pracach matematycznych również nie jest wymagany. Można w nim przedstawić np. jakiś dłuższy dowód, który z pewnych przyczyn pominęliśmy we właściwej części pracy lub (np. w przypadku prac statystycznych) umieścić dane, które analizowaliśmy.

    %%%%%%%%%%%%%%%%%%%%%%%%%%%%%%%%%%%%%%%%%%%%%%%%%%%%%%%%%
    % BIBLIOGRAFIA
    % W tworzeniu bibliografii najlepiej korzystać z BibTex'a,
    % który jest częścią systemu Tex. W naszym przypadku funkcję
    % przechowalni literatury, do której się odwołujemy, pełni
    % plik bibliografia.bib. Nie musimy ręcznie dodawać nowych
    % pozycji do bibliografii. Możemy wejść np. na stronę
    % https://mathscinet.ams.org/mathscinet/index.html,
    % znaleźć odpowiednią pozycję, wybrać ją, a następnie zmienić
    % 'Select alternative format' na BibTeX, skopiować uzyskany
    % tekst, wkleić do pliku bibliografia.bib i skompilować.
    % Gotowe informacje do pliku bibliografia.bib można znaleźć
    % także na https://arxiv.org - gdy znajdziemy interesującą nas
    % pracę, szukamy 'References & Citations' i klikamy 'NASA ADS',
    % a potem 'Bibtex entry for this abstract'
    % i postępujemy tak jak wcześniej.
    %%%%%%%%%%%%%%%%%%%%%%%%%%%%%%%%%%%%%%%%%%%%%%%%%%%%%%%%%
    \newpage
    % w nawiasie klamrowym wpisujemy nazwę pliku z bibliografią w formacie .bib
    \bibliografia{bibliografia}
\end{document}